\begin{table}[H]
   \caption{Table A} 
   \label{tab:tab-a}
   \small
   \centering
   \begin{tabular}{lccr}
   \toprule\toprule
   \textbf{Header 1} & \textbf{Header 2} & \textbf{Header 3} & \textbf{Header 4} \\ 
   \midrule
   Item 1 & Item 2 & Item 3 & Item 4 \\
   Item 1 & Item 2 & Item 3 & Item 4\\
   \bottomrule
   \end{tabular}
\end{table}

\begin{table}[H]
   \caption{Table B} 
   \label{tab:tab-b}
   \small
   \centering
   \begin{tabular}{lccr}
   \toprule\toprule
   \textbf{Header 1} & \textbf{Header 2} & \textbf{Header 3} & \textbf{Header 4} \\ 
   \midrule
   Item 1 & Item 2 & Item 3 & Item 4 \\
   Item 1 & Item 2 & Item 3 & Item 4\\
   \bottomrule
   \end{tabular}
\end{table}

We can reference a range of tables as seen here: \refrange{tab:tab-a}{tab:tab-b}.
Also note how the "Tables" prefix is automatically added within the document text whenever the range reference is called.